%% LyX 2.3.4.2 created this file.  For more info, see http://www.lyx.org/.
%% Do not edit unless you really know what you are doing.
\documentclass[ruled]{article}
\usepackage{courier}
\usepackage[T1]{fontenc}
\usepackage[latin9]{inputenc}
\usepackage[letterpaper]{geometry}
\geometry{verbose}
\usepackage{color}
\usepackage{algorithm2e}
\usepackage{amsmath}
\usepackage{amssymb}
\usepackage{graphicx}
\usepackage[unicode=true,
 bookmarks=false,
 breaklinks=false,pdfborder={0 0 1},backref=section,colorlinks=true]
 {hyperref}
 
 \usepackage{parskip}
\setlength{\parindent}{0pt}

\usepackage{graphicx}
\usepackage{fancyhdr}
\usepackage[dvipsnames]{xcolor}
\usepackage{ifthen}
\usepackage{titlesec}

\titleformat{\section}
{\color{RoyalPurple}\normalfont\Large\bfseries}
{\color{RoyalPurple}\thesection}{1em}{}

\titleformat{\subsection}
{\color{RoyalPurple}\normalfont\large\bfseries}
{\color{RoyalPurple}\thesubsection}{1em}{}

\makeatletter

%%%%%%%%%%%%%%%%%%%%%%%%%%%%%% LyX specific LaTeX commands.
\providecommand{\LyX}{\texorpdfstring%
  {L\kern-.1667em\lower.25em\hbox{Y}\kern-.125emX\@}
  {LyX}}
%% Special footnote code from the package 'stblftnt.sty'
%% Author: Robin Fairbairns -- Last revised Dec 13 1996
\let\SF@@footnote\footnote
\def\footnote{\ifx\protect\@typeset@protect
    \expandafter\SF@@footnote
  \else
    \expandafter\SF@gobble@opt
  \fi
}
\expandafter\def\csname SF@gobble@opt \endcsname{\@ifnextchar[%]
  \SF@gobble@twobracket
  \@gobble
}
\edef\SF@gobble@opt{\noexpand\protect
  \expandafter\noexpand\csname SF@gobble@opt \endcsname}
\def\SF@gobble@twobracket[#1]#2{}
%% Because html converters don't know tabularnewline
\providecommand{\tabularnewline}{\\}

\@ifundefined{date}{}{\date{}}
%%%%%%%%%%%%%%%%%%%%%%%%%%%%%% User specified LaTeX commands.
\definecolor{mygreen}{rgb}{0,0.6,0}
\definecolor{mygray}{rgb}{0.5,0.5,0.5}
\definecolor{mymauve}{rgb}{0.58,0,0.82}

\makeatother

% macros
\input{macros.tex}
\input{rosenberg-macros.tex}

\newif\ifsolution
% uncomment the following line to hide solutions
\solutionfalse
\ifsolution
\newcommand{\solution}[1]{\textbf{Solution:} #1}
\else
\newcommand{\solution}[1]{}
\fi

\begin{document}


\pagestyle{fancy} \lhead{\includegraphics[width=4cm]{logo.PNG}} \rhead{}

\begin{center}
{\LARGE
DS-GA 1003: Machine Learning (Spring 2020) 
} 
\end{center}
\begin{center}
    {\Large
Homework 6: Multiclass, Trees, Gradient Boosting
}
\end{center}


\begin{minipage}{0.6\textwidth}
 \vspace{2mm}
{\large
{\color{RoyalPurple} Due: Tuesday, May 1, 2020 at 11:59pm} 
} 
\end{minipage}

\vspace{0.4cm}
{\color{RoyalPurple} \bf Instructions.} You should upload your code and plots to Gradescope. Please map the Gradescope entry on the rubric to your name and NetId. You must follow the policies for submission detailed in Homework 0.   

\vspace{0.6cm}
\hrule
\vspace{0.6cm}



\section{SGD for Multiclass Linear SVM}

Suppose our output space and our action space are given as follows:
$\cy=\ca=\left\{ 1,\ldots,k\right\} $. Given a non-negative class-sensitive
loss function $\Delta:\cy\times\ca\to[0,\infty)$ and a class-sensitive
feature mapping $\Psi:\cx\times\cy\to\reals^{d}$. Our prediction
function $f:\cx\to\cy$ is given by
\[
f_{w}(x)=\argmax_{y\in\cy}\left\langle w,\Psi(x,y)\right\rangle 
\]
For training data $(x_{1},y_{1}),\ldots,(x_{n},y_{n})\in\cx\times\cy$,
let $J(w)$ be the $\ell_{2}$-regularized empirical risk function
for the multiclass hinge loss. We can write this as
\[
J(w)=\lambda\|w\|^{2}+\frac{1}{n}\sum_{i=1}^{n}\max_{y\in\cy}\left[\Delta\left(y_{i},y\right)+\left\langle w,\Psi(x_{i},y)-\Psi(x_{i},y_{i})\right\rangle \right],
\]
for some $\lambda>0$.
\begin{enumerate}
\item Show that $J(w)$ is a convex function of $w$. You
may use any of the rules about convex functions described in our \href{https://davidrosenberg.github.io/mlcourse/Notes/convex-optimization.pdf}{notes on Convex Optimization},
in previous assignments, or in the Boyd and Vandenberghe book, though
you should cite the general facts you are using. {[}Hint: If $f_{1},\ldots,f_{m}:\reals^{n}\to\reals$
are convex, then their pointwise maximum $f(x)=\max\left\{ f_{1}(x),\ldots,f_{m}(x)\right\} $
is also convex.{]}


\item Since $J(w)$ is convex, it has a subgradient at every point. Give
an expression for a subgradient of $J(w)$. You may use any standard
results about subgradients, including the result from an earlier homework
about subgradients of the pointwise maxima of functions. (Hint: It
may be helpful to refer to $\hat{y}_{i}=\argmax_{y\in\cy}\left[\Delta\left(y_{i},y\right)+\left\langle w,\Psi(x_{i},y)-\Psi(x_{i},y_{i})\right\rangle \right]$.)


\item Give an expression for the stochastic subgradient based on the point
$(x_{i},y_{i})$.


\item Give an expression for a minibatch subgradient, based on the points
$(x_{i},y_{i}),\ldots,\left(x_{i+m-1},y_{i+m-1}\right)$.

\end{enumerate}

\section{{[}Optional{]} Hinge Loss is a Special Case of Generalized Hinge
Loss}

Let $\cy=\left\{ -1,1\right\} $. Let $\Delta(y,\hat{y})=\ind{y\neq\hat{y}}.$
If $g(x)$ is the score function in our binary classification setting,
then define our compatibility function as 
\begin{eqnarray*}
h(x,1) & = & g(x)/2\\
h(x,-1) & = & -g(x)/2.
\end{eqnarray*}
Show that for this choice of $h$, the multiclass hinge loss reduces
to hinge loss: 
\[
\ell\left(h,\left(x,y\right)\right)=\max_{y'\in\cy}\left[\Delta\left(y,y')\right)+h(x,y')-h(x,y)\right]=\max\left\{ 0,1-yg(x)\right\} 
\]


\section{Multiclass Classification - Implementation}

In this problem we will work on a simple three-class classification
example.
The data is generated and plotted for you in the skeleton code. 

\subsection{One-vs-All (also known as One-vs-Rest)}

In this problem we will implement one-vs-all multiclass classification.
Our approach will assume we have a binary base classifier that returns
a score, and we will predict the class that has the highest score. 
\begin{enumerate}
\item Complete the class OneVsAllClassifier in the skeleton code. Following
the OneVsAllClassifier code is a cell that extracts the results of
the fit and plots the decision region. Include these results in your
submission.
\end{enumerate}


\subsection{Multiclass SVM}

In this question, we will implement stochastic subgradient descent
for the linear multiclass SVM, as described in lecture and in this
problem set. We will use the class-sensitive feature mapping approach
with the ``multivector construction'', as described in our \href{https://github.com/cp71/DS-GA-1003-SPRING-2020-PUBLIC/blob/master/lecture/lec11/lec11-multiclass.pdf}{multiclass classification lecture} (slide 24).
\begin{enumerate}
\item Complete the skeleton code for multiclass SVM. Following the multiclass
SVM implementation, we have included another block of test code. Make
sure to include the results from these tests in your assignment, along
with your code. 
\end{enumerate}


\section{Decision Tree Implementation}

In this problem we'll implement decision trees for both classification
and regression. The strategy will be to implement a generic class,
called \code{Decision\_Tree}, which we'll supply with the loss function
we want to use to make node splitting decisions, as well as the estimator
we'll use to come up with the prediction associated with each leaf
node. For classification, this prediction could be a vector of probabilities,
but for simplicity we'll just consider hard classifications here.
We'll work with the classification and regression data sets from previous
assignments.
\begin{enumerate}
\item Complete the class \code{Decision\_Tree}, given in
the skeleton code. The intended implementation is as follows: Each
object of type \code{Decision\_Tree} represents a single node of
the tree. The depth of that node is represented by the variable self.depth,
with the root node having depth 0. The main job of the fit function
is to decide, given the data provided, how to split the node or whether
it should remain a leaf node. If the node will split, then the splitting
feature and splitting value are recorded, and the left and right subtrees
are fit on the relevant portions of the data. Thus tree-building is
a recursive procedure. We should have as many\code{Decision\_Tree}
objects as there are nodes in the tree. We will not implement pruning\textbf{
}here. Some additional details are given in the skeleton code. 
\item Complete either the \code{compute\_entropy} or \code{compute\_gini}
functions. Run the code provided that builds trees for the two-dimensional
classification data. Include the results. For debugging, you may want
to compare results with sklearn's decision tree. For visualization,
you'll need to install \code{graphviz}.
\item {[}Optional{]} Complete the function \code{mean\_absolute\_deviation\_around\_median}
(MAE). Use the code provided to fit the \code{Regression\_Tree} to
the krr dataset using both the MAE loss and median predictions. Include
the plots for the 6 fits.
\end{enumerate}

\section{Gradient Boosting Machines}

Recall the general gradient boosting algorithm\footnote{Besides the lecture slides, you can find an accessible discussion
of this approach in \url{http://www.saedsayad.com/docs/gbm2.pdf},
in one of the original references \url{http://statweb.stanford.edu/~jhf/ftp/trebst.pdf},
and in this review paper \url{http://web.stanford.edu/~hastie/Papers/buehlmann.pdf}. }, for a given loss function $\ell$ and a hypothesis space $\cf$
of regression functions (i.e. functions mapping from the input space
to $\reals$): 
\begin{enumerate}
\item Initialize $f_{0}(x)=0$. 
\item For $m=1$ to $M$:

\begin{enumerate}
\item Compute: 
\[
{\bf g}_{m}=\left(\left.\frac{\partial}{\partial f(x_{j})}\sum_{i=1}^{n}\ell\left(y_{i},f(x_{i})\right)\right|_{f(x_{i})=f_{m-1}(x_{i}),\,i=1,\ldots,n}\right)_{j=1}^{n}
\]
\item Fit regression model to $-{\bf g}_{m}$: 
\[
h_{m}=\argmin_{h\in\cf}\sum_{i=1}^{n}\left(\left(-{\bf g}_{m}\right)_{i}-h(x_{i})\right)^{2}.
\]
\item Choose fixed step size $\nu_{m}=\nu\in(0,1]$, or take 
\[
\nu_{m}=\argmin_{\nu>0}\sum_{i=1}^{n}\ell\left(y_{i},f_{m-1}(x_{i})+\nu h_{m}(x_{i})\right).
\]
\item Take the step: 
\[
f_{m}(x)=f_{m-1}(x)+\nu_{m}h_{m}(x)
\]
\end{enumerate}
\item Return $f_{M}$. 
\end{enumerate}
In this problem we'll derive a special case of the general gradient
boosting framework: BinomialBoost. 
\begin{enumerate}

\item Let's consider the classification framework, where $\cy=\left\{ -1,1\right\} $.
In lecture, we noted that AdaBoost corresponds to forward stagewise
additive modeling with the exponential loss, and that the exponential
loss is not very robust to outliers (i.e. outliers can have a large
effect on the final prediction function). Instead, let's consider
the logistic loss 
\[
\ell(m)=\ln\left(1+e^{-m}\right),
\]
where $m=yf(x)$ is the margin. Similar to what we did in the $\ell_{2}$-Boosting
question, write an expression for $h_{m}$ as an argmin over $\cf$.

\end{enumerate}


\section{Gradient Boosting Implementation}

This method goes by many names, including gradient boosting machines
(GBM), generalized boosting models (GBM), AnyBoost, and gradient boosted
regression trees (GBRT), among others. Although one of the nice aspects
of gradient boosting is that it can be applied to any problem with
a subdifferentiable loss function, here we'll keep things simple and
consider the standard regression setting with square loss. 
\begin{enumerate}
\item Complete the \code{gradient\_boosting} class. As the base regression
algorithm, you may use sklearn's regression tree. You should use
the square loss for the tree splitting rule and the mean function
for the leaf prediction rule. Run the code provided to build gradient
boosting models on the classification and regression data sets, and
include the plots generated. Note that we are using square loss to
fit the classification data, as well as the regression data.
\item {[}Optional{]} Repeat the previous runs on the classification data
set, but use a different classification loss, such as logistic loss
or hinge loss. Include the new code and plots of your results. Note
that you should still use the same regression tree settings for the
base regression algorithm. 
\end{enumerate}

\end{document}